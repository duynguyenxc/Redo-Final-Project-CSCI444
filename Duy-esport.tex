% Options for packages loaded elsewhere
\PassOptionsToPackage{unicode}{hyperref}
\PassOptionsToPackage{hyphens}{url}
\PassOptionsToPackage{dvipsnames,svgnames,x11names}{xcolor}
%
\documentclass[
  letterpaper,
  DIV=11,
  numbers=noendperiod]{scrartcl}

\usepackage{amsmath,amssymb}
\usepackage{iftex}
\ifPDFTeX
  \usepackage[T1]{fontenc}
  \usepackage[utf8]{inputenc}
  \usepackage{textcomp} % provide euro and other symbols
\else % if luatex or xetex
  \usepackage{unicode-math}
  \defaultfontfeatures{Scale=MatchLowercase}
  \defaultfontfeatures[\rmfamily]{Ligatures=TeX,Scale=1}
\fi
\usepackage{lmodern}
\ifPDFTeX\else  
    % xetex/luatex font selection
\fi
% Use upquote if available, for straight quotes in verbatim environments
\IfFileExists{upquote.sty}{\usepackage{upquote}}{}
\IfFileExists{microtype.sty}{% use microtype if available
  \usepackage[]{microtype}
  \UseMicrotypeSet[protrusion]{basicmath} % disable protrusion for tt fonts
}{}
\makeatletter
\@ifundefined{KOMAClassName}{% if non-KOMA class
  \IfFileExists{parskip.sty}{%
    \usepackage{parskip}
  }{% else
    \setlength{\parindent}{0pt}
    \setlength{\parskip}{6pt plus 2pt minus 1pt}}
}{% if KOMA class
  \KOMAoptions{parskip=half}}
\makeatother
\usepackage{xcolor}
\setlength{\emergencystretch}{3em} % prevent overfull lines
\setcounter{secnumdepth}{-\maxdimen} % remove section numbering
% Make \paragraph and \subparagraph free-standing
\makeatletter
\ifx\paragraph\undefined\else
  \let\oldparagraph\paragraph
  \renewcommand{\paragraph}{
    \@ifstar
      \xxxParagraphStar
      \xxxParagraphNoStar
  }
  \newcommand{\xxxParagraphStar}[1]{\oldparagraph*{#1}\mbox{}}
  \newcommand{\xxxParagraphNoStar}[1]{\oldparagraph{#1}\mbox{}}
\fi
\ifx\subparagraph\undefined\else
  \let\oldsubparagraph\subparagraph
  \renewcommand{\subparagraph}{
    \@ifstar
      \xxxSubParagraphStar
      \xxxSubParagraphNoStar
  }
  \newcommand{\xxxSubParagraphStar}[1]{\oldsubparagraph*{#1}\mbox{}}
  \newcommand{\xxxSubParagraphNoStar}[1]{\oldsubparagraph{#1}\mbox{}}
\fi
\makeatother

\usepackage{color}
\usepackage{fancyvrb}
\newcommand{\VerbBar}{|}
\newcommand{\VERB}{\Verb[commandchars=\\\{\}]}
\DefineVerbatimEnvironment{Highlighting}{Verbatim}{commandchars=\\\{\}}
% Add ',fontsize=\small' for more characters per line
\usepackage{framed}
\definecolor{shadecolor}{RGB}{241,243,245}
\newenvironment{Shaded}{\begin{snugshade}}{\end{snugshade}}
\newcommand{\AlertTok}[1]{\textcolor[rgb]{0.68,0.00,0.00}{#1}}
\newcommand{\AnnotationTok}[1]{\textcolor[rgb]{0.37,0.37,0.37}{#1}}
\newcommand{\AttributeTok}[1]{\textcolor[rgb]{0.40,0.45,0.13}{#1}}
\newcommand{\BaseNTok}[1]{\textcolor[rgb]{0.68,0.00,0.00}{#1}}
\newcommand{\BuiltInTok}[1]{\textcolor[rgb]{0.00,0.23,0.31}{#1}}
\newcommand{\CharTok}[1]{\textcolor[rgb]{0.13,0.47,0.30}{#1}}
\newcommand{\CommentTok}[1]{\textcolor[rgb]{0.37,0.37,0.37}{#1}}
\newcommand{\CommentVarTok}[1]{\textcolor[rgb]{0.37,0.37,0.37}{\textit{#1}}}
\newcommand{\ConstantTok}[1]{\textcolor[rgb]{0.56,0.35,0.01}{#1}}
\newcommand{\ControlFlowTok}[1]{\textcolor[rgb]{0.00,0.23,0.31}{\textbf{#1}}}
\newcommand{\DataTypeTok}[1]{\textcolor[rgb]{0.68,0.00,0.00}{#1}}
\newcommand{\DecValTok}[1]{\textcolor[rgb]{0.68,0.00,0.00}{#1}}
\newcommand{\DocumentationTok}[1]{\textcolor[rgb]{0.37,0.37,0.37}{\textit{#1}}}
\newcommand{\ErrorTok}[1]{\textcolor[rgb]{0.68,0.00,0.00}{#1}}
\newcommand{\ExtensionTok}[1]{\textcolor[rgb]{0.00,0.23,0.31}{#1}}
\newcommand{\FloatTok}[1]{\textcolor[rgb]{0.68,0.00,0.00}{#1}}
\newcommand{\FunctionTok}[1]{\textcolor[rgb]{0.28,0.35,0.67}{#1}}
\newcommand{\ImportTok}[1]{\textcolor[rgb]{0.00,0.46,0.62}{#1}}
\newcommand{\InformationTok}[1]{\textcolor[rgb]{0.37,0.37,0.37}{#1}}
\newcommand{\KeywordTok}[1]{\textcolor[rgb]{0.00,0.23,0.31}{\textbf{#1}}}
\newcommand{\NormalTok}[1]{\textcolor[rgb]{0.00,0.23,0.31}{#1}}
\newcommand{\OperatorTok}[1]{\textcolor[rgb]{0.37,0.37,0.37}{#1}}
\newcommand{\OtherTok}[1]{\textcolor[rgb]{0.00,0.23,0.31}{#1}}
\newcommand{\PreprocessorTok}[1]{\textcolor[rgb]{0.68,0.00,0.00}{#1}}
\newcommand{\RegionMarkerTok}[1]{\textcolor[rgb]{0.00,0.23,0.31}{#1}}
\newcommand{\SpecialCharTok}[1]{\textcolor[rgb]{0.37,0.37,0.37}{#1}}
\newcommand{\SpecialStringTok}[1]{\textcolor[rgb]{0.13,0.47,0.30}{#1}}
\newcommand{\StringTok}[1]{\textcolor[rgb]{0.13,0.47,0.30}{#1}}
\newcommand{\VariableTok}[1]{\textcolor[rgb]{0.07,0.07,0.07}{#1}}
\newcommand{\VerbatimStringTok}[1]{\textcolor[rgb]{0.13,0.47,0.30}{#1}}
\newcommand{\WarningTok}[1]{\textcolor[rgb]{0.37,0.37,0.37}{\textit{#1}}}

\providecommand{\tightlist}{%
  \setlength{\itemsep}{0pt}\setlength{\parskip}{0pt}}\usepackage{longtable,booktabs,array}
\usepackage{calc} % for calculating minipage widths
% Correct order of tables after \paragraph or \subparagraph
\usepackage{etoolbox}
\makeatletter
\patchcmd\longtable{\par}{\if@noskipsec\mbox{}\fi\par}{}{}
\makeatother
% Allow footnotes in longtable head/foot
\IfFileExists{footnotehyper.sty}{\usepackage{footnotehyper}}{\usepackage{footnote}}
\makesavenoteenv{longtable}
\usepackage{graphicx}
\makeatletter
\newsavebox\pandoc@box
\newcommand*\pandocbounded[1]{% scales image to fit in text height/width
  \sbox\pandoc@box{#1}%
  \Gscale@div\@tempa{\textheight}{\dimexpr\ht\pandoc@box+\dp\pandoc@box\relax}%
  \Gscale@div\@tempb{\linewidth}{\wd\pandoc@box}%
  \ifdim\@tempb\p@<\@tempa\p@\let\@tempa\@tempb\fi% select the smaller of both
  \ifdim\@tempa\p@<\p@\scalebox{\@tempa}{\usebox\pandoc@box}%
  \else\usebox{\pandoc@box}%
  \fi%
}
% Set default figure placement to htbp
\def\fps@figure{htbp}
\makeatother

\usepackage{booktabs}
\usepackage{longtable}
\usepackage{array}
\usepackage{multirow}
\usepackage{wrapfig}
\usepackage{float}
\usepackage{colortbl}
\usepackage{pdflscape}
\usepackage{tabu}
\usepackage{threeparttable}
\usepackage{threeparttablex}
\usepackage[normalem]{ulem}
\usepackage{makecell}
\usepackage{xcolor}
\KOMAoption{captions}{tableheading}
\makeatletter
\@ifpackageloaded{caption}{}{\usepackage{caption}}
\AtBeginDocument{%
\ifdefined\contentsname
  \renewcommand*\contentsname{Table of contents}
\else
  \newcommand\contentsname{Table of contents}
\fi
\ifdefined\listfigurename
  \renewcommand*\listfigurename{List of Figures}
\else
  \newcommand\listfigurename{List of Figures}
\fi
\ifdefined\listtablename
  \renewcommand*\listtablename{List of Tables}
\else
  \newcommand\listtablename{List of Tables}
\fi
\ifdefined\figurename
  \renewcommand*\figurename{Figure}
\else
  \newcommand\figurename{Figure}
\fi
\ifdefined\tablename
  \renewcommand*\tablename{Table}
\else
  \newcommand\tablename{Table}
\fi
}
\@ifpackageloaded{float}{}{\usepackage{float}}
\floatstyle{ruled}
\@ifundefined{c@chapter}{\newfloat{codelisting}{h}{lop}}{\newfloat{codelisting}{h}{lop}[chapter]}
\floatname{codelisting}{Listing}
\newcommand*\listoflistings{\listof{codelisting}{List of Listings}}
\makeatother
\makeatletter
\makeatother
\makeatletter
\@ifpackageloaded{caption}{}{\usepackage{caption}}
\@ifpackageloaded{subcaption}{}{\usepackage{subcaption}}
\makeatother

\usepackage{bookmark}

\IfFileExists{xurl.sty}{\usepackage{xurl}}{} % add URL line breaks if available
\urlstyle{same} % disable monospaced font for URLs
\hypersetup{
  pdftitle={Group 9 -- Esports vs Software Engineers},
  pdfauthor={Duy Nguyen -- Esports Section; Bat-Orgil Erdenebat -- Software Engineer Salary Section; Daniel Jimenez -- Software Engineer Career / Experience Section},
  colorlinks=true,
  linkcolor={blue},
  filecolor={Maroon},
  citecolor={Blue},
  urlcolor={Blue},
  pdfcreator={LaTeX via pandoc}}


\title{Group 9 -- Esports vs Software Engineers}
\author{Duy Nguyen -- Esports Section \and Bat-Orgil Erdenebat --
Software Engineer Salary Section \and Daniel Jimenez -- Software
Engineer Career / Experience Section}
\date{2025-12-04}

\begin{document}
\maketitle

\renewcommand*\contentsname{Table of contents}
{
\hypersetup{linkcolor=}
\setcounter{tocdepth}{3}
\tableofcontents
}

\subsection{Question 1 -- How is prize money distributed among the top
1000 esports
players?}\label{question-1-how-is-prize-money-distributed-among-the-top-1000-esports-players}

My first question I decided to look directly at prize money. I want to
understand three things:

1.What is a ``typical'' total earning for a player in the top 1000?

2.How unequal is the prize money inside this group, are there just a few
people far above the rest?

3.Who are those top earners, and which game do they rely on?

To answer that, I start with a summary table, then a histogram for the
whole group, and finally I zoom in on the top 5 players and on the first
part of the ranking curve.

1.1 Summary of total earnings for the top 1000 players

\begin{Shaded}
\begin{Highlighting}[]
\NormalTok{pacman}\SpecialCharTok{::}\FunctionTok{p\_load}\NormalTok{(tidyverse,plotly,kableExtra,tinytex)}

\FunctionTok{theme\_set}\NormalTok{(}\FunctionTok{theme\_minimal}\NormalTok{())}

\NormalTok{players}\OtherTok{\textless{}{-}} \FunctionTok{read\_csv}\NormalTok{(}\StringTok{"data/players\_top1000\_clean\_12\_4.csv"}\NormalTok{)}
\NormalTok{games }\OtherTok{\textless{}{-}}\FunctionTok{read\_csv}\NormalTok{(}\StringTok{"data/general\_esport\_games\_clean.csv"}\NormalTok{)}


\NormalTok{q1\_summary }\OtherTok{\textless{}{-}}\NormalTok{ players }\SpecialCharTok{\%\textgreater{}\%}
  \FunctionTok{summarise}\NormalTok{(}\AttributeTok{n\_players =} \FunctionTok{n}\NormalTok{(),}
            \AttributeTok{mean\_earnings=} \FunctionTok{mean}\NormalTok{(total\_overall\_usd, }\AttributeTok{na.rm =} \ConstantTok{TRUE}\NormalTok{),}
            \AttributeTok{median\_earnings=} \FunctionTok{median}\NormalTok{(total\_overall\_usd, }\AttributeTok{na.rm=} \ConstantTok{TRUE}\NormalTok{),}
            \AttributeTok{min\_earnings =} \FunctionTok{min}\NormalTok{(total\_overall\_usd, }\AttributeTok{na.rm =}\ConstantTok{TRUE}\NormalTok{),}
            \AttributeTok{max\_earnings =} \FunctionTok{max}\NormalTok{(total\_overall\_usd, }\AttributeTok{na.rm =}\ConstantTok{TRUE}\NormalTok{))}

\NormalTok{q1\_summary}
\end{Highlighting}
\end{Shaded}

\begin{verbatim}
# A tibble: 1 x 5
  n_players mean_earnings median_earnings min_earnings max_earnings
      <int>         <dbl>           <dbl>        <dbl>        <dbl>
1      1000       836795.         550481.       332275     7184163.
\end{verbatim}

The summary table confirms that we have 1,000 players in this sample.
The average player has earned about \$0.84 million in prize money, but
the median is lower at roughly \$0.55 million. The last person in the
top 1000 still has more than \$0.33 million, while the number-one player
is above \$7 million. The fact that the mean is well above the median
already tells us that the distribution is skewed to the right: a small
group of very rich players is pulling the average up above what most
players actually earn.

1.2 Histogram -- distribution of total prize money

\begin{Shaded}
\begin{Highlighting}[]
\NormalTok{players }\SpecialCharTok{\%\textgreater{}\%}
  \FunctionTok{mutate}\NormalTok{(}\AttributeTok{total\_overall\_m =}\NormalTok{ total\_overall\_usd }\SpecialCharTok{/} \FloatTok{1e6}\NormalTok{) }\SpecialCharTok{\%\textgreater{}\%}
  \FunctionTok{ggplot}\NormalTok{(}\FunctionTok{aes}\NormalTok{(}\AttributeTok{x =}\NormalTok{ total\_overall\_m)) }\SpecialCharTok{+}
  \FunctionTok{geom\_histogram}\NormalTok{(}\AttributeTok{binwidth =} \FloatTok{0.25}\NormalTok{, }\AttributeTok{color =} \StringTok{"white"}\NormalTok{, }\AttributeTok{fill =} \StringTok{"\#1f78b4"}\NormalTok{)}\SpecialCharTok{+}
  \FunctionTok{scale\_x\_continuous}\NormalTok{(}\AttributeTok{breaks =} \DecValTok{0}\SpecialCharTok{:}\DecValTok{7}\NormalTok{)}\SpecialCharTok{+}
  \FunctionTok{labs}\NormalTok{(}\AttributeTok{title =} \StringTok{"Distribution of total earnings (top 1000 esports players)"}\NormalTok{,}
       \AttributeTok{x =} \StringTok{"Total prize money (million USD)"}\NormalTok{,}
       \AttributeTok{y =} \StringTok{"Number of players"}\NormalTok{)}
\end{Highlighting}
\end{Shaded}

\pandocbounded{\includegraphics[keepaspectratio]{Duy-esport_files/figure-pdf/q1_hist_earnings-1.pdf}}

The histogram makes this skew obvious. Most players are squeezed into
the range from about \$0.3M to \$1M, and that is where the tallest bars
are. This is the ``typical'' zone for top-1000 players: they have made
hundreds of thousands of dollars, but not multiple millions. To the
right, the bars become very short. Above \$2M there are only a few
players in each bin, and between \$3M and \$7M we see just a handful of
cases. The single bar near \$7M corresponds to the very top player. High
prize money exists, but even inside the top 1000 it is rare.

1.3 Top 5 outlier players by total earnings

\begin{Shaded}
\begin{Highlighting}[]
\NormalTok{q1\_top5\_players }\OtherTok{\textless{}{-}}\NormalTok{ players}\SpecialCharTok{\%\textgreater{}\%}
  \FunctionTok{arrange}\NormalTok{(}\FunctionTok{desc}\NormalTok{(total\_overall\_usd)) }\SpecialCharTok{\%\textgreater{}\%}
  \FunctionTok{slice\_head}\NormalTok{(}\AttributeTok{n =} \DecValTok{5}\NormalTok{) }\SpecialCharTok{\%\textgreater{}\%}
  \FunctionTok{mutate}\NormalTok{(}\AttributeTok{total\_overall\_m =} \FunctionTok{round}\NormalTok{(total\_overall\_usd }\SpecialCharTok{/} \FloatTok{1e6}\NormalTok{, }\DecValTok{2}\NormalTok{)) }\SpecialCharTok{\%\textgreater{}\%}
  \FunctionTok{select}\NormalTok{(rank,player\_name,country,highest\_paying\_game,total\_overall\_m,pct\_of\_total\_num)}

\NormalTok{q1\_top5\_players}\SpecialCharTok{\%\textgreater{}\%}
  \FunctionTok{kbl}\NormalTok{(}\AttributeTok{col.names =} \FunctionTok{c}\NormalTok{(}\StringTok{"Rank"}\NormalTok{,}\StringTok{"Player"}\NormalTok{,}\StringTok{"Country"}\NormalTok{,}\StringTok{"Main Game"}\NormalTok{,}\StringTok{"Total earnings (M USD)"}\NormalTok{,}\StringTok{"Game \% of total"}\NormalTok{),}
      \AttributeTok{caption =} \StringTok{"Top 5 highest{-}earning esports players and their main game."}\NormalTok{) }\SpecialCharTok{\%\textgreater{}\%}
  \FunctionTok{kable\_paper}\NormalTok{(}\AttributeTok{full\_width =} \ConstantTok{FALSE}\NormalTok{) }\SpecialCharTok{\%\textgreater{}\%}
  \FunctionTok{kable\_styling}\NormalTok{(}\AttributeTok{position =} \StringTok{"center"}\NormalTok{) }\SpecialCharTok{\%\textgreater{}\%}
  \FunctionTok{column\_spec}\NormalTok{(}\DecValTok{5}\NormalTok{, }\AttributeTok{bold =} \ConstantTok{TRUE}\NormalTok{)}
\end{Highlighting}
\end{Shaded}

\begin{table}
\centering\centering
\caption{Top 5 highest-earning esports players and their main game.}
\centering
\begin{tabular}[t]{r|l|l|l|>{}r|r}
\hline
Rank & Player & Country & Main Game & Total earnings (M USD) & Game \% of total\\
\hline
1 & Johan Sundstein & Denmark & Dota 2 & \textbf{7.18} & 99.84\\
\hline
2 & Jesse Vainikka & Finland & Dota 2 & \textbf{6.49} & 99.99\\
\hline
3 & Yaroslav Naidenov & Russian Federation & Dota 2 & \textbf{6.23} & 100.00\\
\hline
4 & Anathan Pham & Australia & Dota 2 & \textbf{6.02} & 100.00\\
\hline
5 & Sébastien Debs & France & Dota 2 & \textbf{5.95} & 100.00\\
\hline
\end{tabular}
\end{table}

The table shows who these extreme outliers are. All five have Dota 2 as
their main game, but they come from different countries. Each of them
has earned between about \$5.9M and \$7.2M. Compared to the median of
roughly \$0.55M, every one of them has more than ten times the prize
money of a ``typical'' player in this group. The last column also shows
that almost all of their income comes from a single title, which
suggests that reaching the very top usually means specialising hard in
one game.

1.4 Plotly Prize-money rank vs total earnings (by game group)

\begin{Shaded}
\begin{Highlighting}[]
\NormalTok{top5\_games }\OtherTok{\textless{}{-}}\NormalTok{ players }\SpecialCharTok{\%\textgreater{}\%}
  \FunctionTok{count}\NormalTok{(highest\_paying\_game, }\AttributeTok{sort =} \ConstantTok{TRUE}\NormalTok{)}\SpecialCharTok{\%\textgreater{}\%}
  \FunctionTok{slice\_head}\NormalTok{(}\AttributeTok{n =} \DecValTok{5}\NormalTok{) }\SpecialCharTok{\%\textgreater{}\%}
  \FunctionTok{pull}\NormalTok{(highest\_paying\_game)}

\NormalTok{players\_plot }\OtherTok{\textless{}{-}}\NormalTok{ players}\SpecialCharTok{\%\textgreater{}\%}
  \FunctionTok{mutate}\NormalTok{(}\AttributeTok{total\_overall\_m =}\NormalTok{ total\_overall\_usd}\SpecialCharTok{/} \FloatTok{1e6}\NormalTok{,}
         \AttributeTok{main\_game\_group =} \FunctionTok{if\_else}\NormalTok{(highest\_paying\_game }\SpecialCharTok{\%in\%}\NormalTok{ top5\_games,highest\_paying\_game,}\StringTok{"Other games"}\NormalTok{))}

\NormalTok{q1\_plotly }\OtherTok{\textless{}{-}} \FunctionTok{plot\_ly}\NormalTok{(}\AttributeTok{data=}\NormalTok{ players\_plot,}
                     \AttributeTok{x =} \SpecialCharTok{\textasciitilde{}}\NormalTok{rank,}\AttributeTok{y =} \SpecialCharTok{\textasciitilde{}}\NormalTok{total\_overall\_m,}\AttributeTok{color =} \SpecialCharTok{\textasciitilde{}}\NormalTok{main\_game\_group,}
                     \AttributeTok{type =} \StringTok{"scatter"}\NormalTok{,}\AttributeTok{mode =} \StringTok{"markers"}\NormalTok{,}\AttributeTok{hoverinfo =} \StringTok{"text"}\NormalTok{,}
                     \AttributeTok{text =} \SpecialCharTok{\textasciitilde{}}\FunctionTok{paste0}\NormalTok{(}\StringTok{"Rank (prize money): "}\NormalTok{, rank,}
                                    \StringTok{"\textless{}br\textgreater{}Player: "}\NormalTok{, player\_name,}
                                    \StringTok{"\textless{}br\textgreater{}Country: "}\NormalTok{, country,}
                                    \StringTok{"\textless{}br\textgreater{}Game: "}\NormalTok{, highest\_paying\_game,}
                                    \StringTok{"\textless{}br\textgreater{}Total earnings: $"}\NormalTok{, }\FunctionTok{round}\NormalTok{(total\_overall\_usd, }\DecValTok{0}\NormalTok{)))}\SpecialCharTok{\%\textgreater{}\%}
  \FunctionTok{layout}\NormalTok{(}\AttributeTok{title =} \StringTok{"Rank vs total earnings (top 1000 esports players)"}\NormalTok{,}
         \AttributeTok{xaxis =} \FunctionTok{list}\NormalTok{(}\AttributeTok{title =} \StringTok{"Prize{-}money rank (1 = highest earner)"}\NormalTok{,}
                      \AttributeTok{autorange =} \StringTok{"reversed"}\NormalTok{,}
                      \AttributeTok{rangeslider =} \FunctionTok{list}\NormalTok{(}\AttributeTok{visible =} \ConstantTok{TRUE}\NormalTok{)),}
         \AttributeTok{yaxis =} \FunctionTok{list}\NormalTok{(}\AttributeTok{title =} \StringTok{"Total prize money (million USD)"}\NormalTok{),}
         \AttributeTok{legend =} \FunctionTok{list}\NormalTok{(}\AttributeTok{title =} \FunctionTok{list}\NormalTok{(}\AttributeTok{text =} \StringTok{"Main game group\textless{}br\textgreater{}(5 most common games in top 1000)"}\NormalTok{)))}

\NormalTok{q1\_plotly}
\end{Highlighting}
\end{Shaded}

\pandocbounded{\includegraphics[keepaspectratio]{Duy-esport_files/figure-pdf/q1_plotly_rank-1.pdf}}

In the interactive rank plot, the x-axis orders players by prize-money
rank (1 = highest earner) and the y-axis shows total prize money in
millions of USD. The colours mark game groups: the five most common main
games in the top 1000, plus an ``Other games'' group.

If we zoom in on the top 100 ranks, the curve is fairly flat between
about 100 and 40 (most players between 1.5 and a bit over 2 million
USD). From around rank 30 upward, the curve bends sharply, and the top
10 jump even further away, ending at over 7 million USD for rank 1.

Looking at the colours, the very top part of the curve is dominated by
the Dota 2 group, with only a few points from other games mixed in. In
contrast, the ``Other games'' group appears more often in the
lower-earning parts of the ranking.

Overall, Question 1 shows that prize money among the top 1000 esports
players is very unequal. A typical player in this elite group earns
around half a million dollars in total, but a small cluster of Dota 2
specialists earn between six and seven million. Most players are bunched
together in the lower range, and then the curve jumps suddenly for the
top 20--30 names. This unequal, ``top-heavy'' pattern is important
context when we later compare esports prize money with the more stable
salaries of software engineers in the rest of the project.

\subsection{Question 2 -- The Source of Income Risk (Game-Level
Look)}\label{question-2-the-source-of-income-risk-game-level-look}

To what extent does the prize money structure of top esports games show
a built-in risk (volatility) for professional players?

Question 1 showed that Dota 2 has the top earners. Now we need to check
how that money structure works. My goal is to prove that esports money,
even for the best, is not steady and carries a high risk, very different
from a Software Engineer's steady salary. The scatter plot below
compares Total Prize Money (Y-axis) with the Percent of Money from
Offline Events (X-axis). This X-axis is the key measure for income focus
and instability.

\begin{Shaded}
\begin{Highlighting}[]
\NormalTok{top5\_earnings\_games }\OtherTok{\textless{}{-}}\NormalTok{ games}\SpecialCharTok{\%\textgreater{}\%}
  \FunctionTok{arrange}\NormalTok{(}\FunctionTok{desc}\NormalTok{(total\_earnings\_usd))}\SpecialCharTok{\%\textgreater{}\%}
  \FunctionTok{slice\_head}\NormalTok{(}\AttributeTok{n=}\DecValTok{5}\NormalTok{)}\SpecialCharTok{\%\textgreater{}\%}
  \FunctionTok{pull}\NormalTok{(game)}

\NormalTok{games\_plot\_fixed }\OtherTok{\textless{}{-}}\NormalTok{games}\SpecialCharTok{\%\textgreater{}\%}
  \FunctionTok{filter}\NormalTok{(total\_earnings\_usd}\SpecialCharTok{\textgreater{}} \DecValTok{1000000}\NormalTok{)}\SpecialCharTok{\%\textgreater{}\%}     
  \FunctionTok{mutate}\NormalTok{(}\AttributeTok{total\_earnings\_m=}\NormalTok{ total\_earnings\_usd }\SpecialCharTok{/} \FloatTok{1e6}\NormalTok{,}
         \AttributeTok{game\_group=} \FunctionTok{if\_else}\NormalTok{(game }\SpecialCharTok{\%in\%}\NormalTok{ top5\_earnings\_games, game,}\StringTok{"Other games"}\NormalTok{))}

\NormalTok{q2\_scatter\_fixed}\OtherTok{\textless{}{-}} \FunctionTok{plot\_ly}\NormalTok{(}\AttributeTok{data =}\NormalTok{ games\_plot\_fixed,}\AttributeTok{x =} \SpecialCharTok{\textasciitilde{}}\NormalTok{percent\_offline,}
                           \AttributeTok{y =} \SpecialCharTok{\textasciitilde{}}\NormalTok{total\_earnings\_m,}\AttributeTok{type =} \StringTok{"scatter"}\NormalTok{,}
                           \AttributeTok{mode =} \StringTok{"markers"}\NormalTok{,}\AttributeTok{color =} \SpecialCharTok{\textasciitilde{}}\NormalTok{game\_group,}\AttributeTok{size  =} \SpecialCharTok{\textasciitilde{}}\NormalTok{total\_players,}
                           \AttributeTok{hovertemplate =} \FunctionTok{paste0}\NormalTok{(}\StringTok{"Game: "}\NormalTok{, games\_plot\_fixed}\SpecialCharTok{$}\NormalTok{game,}\StringTok{"\textless{}br\textgreater{}Group: "}\NormalTok{,games\_plot\_fixed}\SpecialCharTok{$}\NormalTok{game\_group,}
                                                  \StringTok{"\textless{}br\textgreater{}Total earnings: $"}\NormalTok{, }\FunctionTok{round}\NormalTok{(games\_plot\_fixed}\SpecialCharTok{$}\NormalTok{total\_earnings\_usd }\SpecialCharTok{/} \FloatTok{1e6}\NormalTok{, }\DecValTok{2}\NormalTok{),}
                                                  \StringTok{"M"}\NormalTok{,}\StringTok{"\textless{}br\textgreater{}\% Offline: \%\{x:.1\%\}"}\NormalTok{,}\StringTok{"\textless{}br\textgreater{}Genre: "}\NormalTok{, games\_plot\_fixed}\SpecialCharTok{$}\NormalTok{genre,}
                                                  \StringTok{"\textless{}extra\textgreater{}\textless{}/extra\textgreater{}"}\NormalTok{))}\SpecialCharTok{\%\textgreater{}\%}
  \FunctionTok{layout}\NormalTok{(}\AttributeTok{title =} \StringTok{"Total earnings vs offline percentage"}\NormalTok{,}
         \AttributeTok{xaxis =} \FunctionTok{list}\NormalTok{(}\AttributeTok{title =} \StringTok{"Percentage of earnings from offline tournaments (\% of total)"}\NormalTok{,}
                      \AttributeTok{tickformat =} \StringTok{".0\%"}\NormalTok{),}
         \AttributeTok{yaxis =} \FunctionTok{list}\NormalTok{(}\AttributeTok{title =} \StringTok{"Total earnings (million USD)"}\NormalTok{),}
         \AttributeTok{legend =} \FunctionTok{list}\NormalTok{(}\AttributeTok{title =} \FunctionTok{list}\NormalTok{(}\AttributeTok{text =} \StringTok{"Game"}\NormalTok{)))}

\NormalTok{q2\_scatter\_fixed}
\end{Highlighting}
\end{Shaded}

\pandocbounded{\includegraphics[keepaspectratio]{Duy-esport_files/figure-pdf/q2_game_volatility_analysis_fixed-1.pdf}}

In this scatterplot, each dot is one game with more than \$1M in total
prize money. The y-axis shows how big the prize pool is, and the x-axis
shows what percent of that money comes from offline tournaments. The
colours highlight five major titles (Dota 2, League of Legends, CS:GO,
Fortnite, Arena of Valor); all smaller games are grouped into ``Other
games''.

Dota 2 stands out in the top-right corner: it has by far the largest
total prize pool (over \$360M) and also one of the highest offline
shares, with about 87\% of its money coming from LAN events. This means
that the game that creates the very highest earners is also the game
where income depends most heavily on a few big offline tournaments,
especially The International.

Other major games sit in different parts of the plot. CS:GO and League
of Legends also have large total earnings, but their offline percentages
are more in the middle range, so prize money is spread across more
events. Fortnite shows the opposite pattern: it has a huge prize pool
but a much lower offline share, because many of its tournaments are
online. Arena of Valor is strongly offline like Dota 2, but its total
prize pool is smaller.

From a risk point of view, games with high offline percentages and big
prize pools create a ``win or starve'' situation. If a team fails to
place well at one or two key LAN events in a year, their prize income
drops sharply. Games with more balanced online/offline splits still have
risk, but the earnings are less tied to a single tournament. In all
cases, this spike-based prize structure is very different from a
software engineer's steady monthly salary, where income does not depend
on winning a specific event.

\subsection{Question 3 -- What is the ``golden age'' for earning prize
money in
esports?}\label{question-3-what-is-the-golden-age-for-earning-prize-money-in-esports}

After looking at how prize money is concentrated (Question 1) and how
risky the game structure is (Question 2), I also want to know when in
life this money is actually earned. Esports is often seen as a ``young
person's career'', so here I use an age-level summary of all recorded
players to see what the age curve really looks like and where the
earning peak happens.

3.1 How many prize-winning players are at each age?

First, I look at how many players with prize money we see at each age
between 13 and 35.

\begin{Shaded}
\begin{Highlighting}[]
\NormalTok{age\_earnings }\OtherTok{\textless{}{-}} \FunctionTok{read\_csv}\NormalTok{(}\StringTok{"data/esports\_earnings\_by\_age.csv"}\NormalTok{)}\SpecialCharTok{\%\textgreater{}\%}
  \FunctionTok{filter}\NormalTok{(age }\SpecialCharTok{\textgreater{}=} \DecValTok{13}\NormalTok{, age }\SpecialCharTok{\textless{}=} \DecValTok{35}\NormalTok{) }\SpecialCharTok{\%\textgreater{}\%}           
  \FunctionTok{mutate}\NormalTok{(}\AttributeTok{total\_prize\_m =}\NormalTok{ total\_prize\_usd }\SpecialCharTok{/} \FloatTok{1e6}\NormalTok{)}\SpecialCharTok{\%\textgreater{}\%}
  \FunctionTok{arrange}\NormalTok{(age)}

\NormalTok{age\_earnings }\SpecialCharTok{\%\textgreater{}\%}
  \FunctionTok{ggplot}\NormalTok{(}\FunctionTok{aes}\NormalTok{(}\AttributeTok{x =}\NormalTok{ age, }\AttributeTok{y =}\NormalTok{ n\_players)) }\SpecialCharTok{+}
  \FunctionTok{geom\_col}\NormalTok{(}\AttributeTok{fill =} \StringTok{"\#1f78b4"}\NormalTok{) }\SpecialCharTok{+}
  \FunctionTok{scale\_x\_continuous}\NormalTok{(}\AttributeTok{breaks =} \FunctionTok{seq}\NormalTok{(}\DecValTok{13}\NormalTok{, }\DecValTok{35}\NormalTok{, }\AttributeTok{by =} \DecValTok{2}\NormalTok{)) }\SpecialCharTok{+}
  \FunctionTok{labs}\NormalTok{(}\AttributeTok{title =} \StringTok{"Number of esports players with prize money by age"}\NormalTok{,}
       \AttributeTok{x =} \StringTok{"Age (years)"}\NormalTok{,}\AttributeTok{y =} \StringTok{"Number of players"}\NormalTok{)}
\end{Highlighting}
\end{Shaded}

\pandocbounded{\includegraphics[keepaspectratio]{Duy-esport_files/figure-pdf/q3_load_age-1.pdf}}

The bars are very small in the early teens and then grow quickly from
about 16 to 20. The tallest bar is at age 20, with almost seven thousand
players who have earned at least some prize money. Ages 19--22 are the
thickest part of the curve -- most active prize-winning players sit in
this narrow band.

After that, the counts shrink. By age 26--27 the numbers have dropped a
lot, and from age 30 onward there are only a few hundred players per
age. In practical terms, most players who ever earn prize money do it in
their late teens and early twenties, and only a small group is still
competing for money in their thirties.

3.2 How is total prize money spread across ages?

The second chart keeps the same ages but now stacks up the total prize
money earned by each age group (in millions of USD).

\begin{Shaded}
\begin{Highlighting}[]
\NormalTok{age\_earnings}\SpecialCharTok{\%\textgreater{}\%}
  \FunctionTok{ggplot}\NormalTok{(}\FunctionTok{aes}\NormalTok{(}\AttributeTok{x =}\NormalTok{ age, }\AttributeTok{y =}\NormalTok{ total\_prize\_m)) }\SpecialCharTok{+}
  \FunctionTok{geom\_col}\NormalTok{(}\AttributeTok{fill =} \StringTok{"\#33a02c"}\NormalTok{) }\SpecialCharTok{+}
  \FunctionTok{scale\_x\_continuous}\NormalTok{(}\AttributeTok{breaks =} \FunctionTok{seq}\NormalTok{(}\DecValTok{13}\NormalTok{, }\DecValTok{35}\NormalTok{, }\AttributeTok{by =} \DecValTok{2}\NormalTok{)) }\SpecialCharTok{+}
  \FunctionTok{labs}\NormalTok{(}\AttributeTok{title =} \StringTok{"Total esports prize money by age"}\NormalTok{,}
       \AttributeTok{x =} \StringTok{"Age (years)"}\NormalTok{,}
       \AttributeTok{y =} \StringTok{"Total prize money (million USD)"}\NormalTok{)}
\end{Highlighting}
\end{Shaded}

\pandocbounded{\includegraphics[keepaspectratio]{Duy-esport_files/figure-pdf/q3_prize_by_age-1.pdf}}

The overall shape is similar, but now we see where the money really
concentrates. Total prize money rises steeply through the late teens and
hits a clear peak at around age 21, where that single age group earns
roughly \$140M in total. Ages 19--23 form a big ``hill'' in the middle
of the plot.

After about 24--25, the bars fall away. By the early thirties, each age
group only contributes a small slice of the total prize pool.

3.3 So what is the ``golden age'' in esports?

Putting the two plots together:

The number of prize-winning players is heavily concentrated between ages
18 and 24.

Total prize money peaks at roughly 20--21, with ages 19, 22 and 23 still
very high.

After the mid-20s, both the number of players and the money drop
quickly, and players over 30 contribute only a tiny part of the overall
prize pool.

Based on this data, the ``golden age'' for earning prize money in
esports is roughly from 18 to 24, with the clearest peak around age 21.
Most of the money that esports players ever make is packed into this
short window. Later in the project, this will contrast with software
engineers, who usually start earning later but can keep high, stable
incomes for many more years.

Overall, esports can create a few very rich players, but it is a
high-risk, short-window path with heavy competition, especially compared
to the more stable and accessible careers in software engineering.




\end{document}
